\section{Introduction}
When we try to make sense of a big amount of information, a natural way 
of beginning is to divide the information into smaller groups, and 
investigate these groups seperately. Often we assign these groups 
manually, but when reviewing a lot of information this can easily become 
unfeasible. To aid us there have been developed a lot of techniques for 
finding clusters in data using different heuristics and techniques to 
obtain groupings. However the automatic methods don't always result in 
meaningful groupings. They often discard information we would 
implicitely include were we grouping the information manually and the 
groups that are calculated might be across features and of sizes that 
render them less useful.

This project aims to explore how to best group a selection of conference 
articles. It is part of a project to provide a tool for conference 
attendees to better get an overview over the hundreds of presentations 
that usually take place at a medium sized scientific conference. If we 
are able to cluster

As part of a project to provide a tool for conference attendees to 
better get an overview of what presentations to attend, The question of 
automatically grouping information in a meaningful way is at the center 
of this project. Working with a project to  with a project called 
Trailhead We have papers about information theory published at a couple 
of conferences, and for each conference it would greatly aid if we could 
automatically organize these papers in groups.

\subsection{The central problem}

There are many methods a selection of scientific articles could 
potentially be grouped, but since a central part of 

\subsection{To Define a Clustering}

\subsection{Terms and Notation}

For the reminder of this paper I will use

\subsection{Structure of the paper}
