\section{Related work}

Over the last 40 years a large variety of algorithms have been invented 
to cluster graphs. Most of these methods were based on different 
measures of an optimal clustering all of which were usually np-hard to 
derive optimal solutions for. This led to various algorithms attacking 
relaxed versions of these problems with different amount of success.  
The field has seen a boost in recent years due to the increased demands 
for handling large data that are usually neatly abstracted as graphs 
\cite[p.  2]{fortunato2010}.  Examples include social networks and the 
world wide web.

In this paper I don't intend to do a thorough review of the field, 
something which many better abled people have already done before 
me\footnote{Examples include \cite{newman2004}, \cite{schaeffer2007} and 
\cite{fortunato2010}}. Instead I have tried to pick out a few distinct 
algorithms that are good enough that each could have been a candidate 
for a clustering algorithm used for grouping the scientific articles.


% von2007 : schaeffer
% fortunato2010 : community detection in graphs
% lancichinetti2008 : comparison
% newman2005 : short and old overview

\subsection{How to measure clustering}

The central difficulty when proposing a new clustering algorithm is to 
measure how it compares with other algorithms. Since there is no agreed 
upon definition on what makes a good cluster, there is naturally no 
definitive candidate for measuring the ``goodness'' of a cluster.  
Historically there has been several different approaches.  
\cite{zachary1977} did a study on the relationships between members of a 
karate club which eventually split into two factions which has since 
then been a benchmark for clustering algorithms. If an algorithm on the 
basis of the social graph can accurately predict the two factions in the 
split, then it has been judged as useful\footnote{See for example 
\cite{girvan2002} and \cite{lancichinetti2009}}.

Fortunately better metrics have been proposed since. One of the most 
used has been the measure of community proposed by \cite{girvan2002} and 
defined as:
\begin{eqnarray}
	Q & = & (\mbox{fraction of edges within communities}) \\
	  & & - (\mbox{expected fraction of such edges}) \\
	& = & \frac{1}{2m} \sum_{i,j} \left[ A_{ij} - \frac{k_i k_j}{2m} \
\right] \delta(c_i, c_j)
\end{eqnarray}
Where $m=\frac{1}{2} \sum_{ij} A_{ij}$, $k_i = \sum_j A_{ij}$ and 
$A_{ij}$ is the weight between the edge connecting $i$ and $j$.  
$\delta(c_i, c_j)$ is the Kroneker Delta which is 1 only if the 
community assignment of $v_i$, $c_i$ is equal to the community 
assignment of $v_j$, $c_j$. The community measure has been the de-facto 
standard but it is not without shortcomings. It has been shown by 
\cite{brandes2007} that the metric doesn't work well with small 
communities.

An alternative measure called Surprise was introduced by 
\cite{arnau2005} and has later been shown to perform better than the 
community measure in a lot of cases by \cite{aldecoa2011}, although it 
has yet to be independently adopted verified and adopted by other 
researchers. The idea behind Surprise is to model the distributions of 
intra- and inter-community links with a cumulative hypergeometric 
distribution. It is defined as follows:
\begin{equation}
	S = -log \sum_{j=p}^{Min(M,n)} \frac{\binom{M}{j} \
	\binom{F-M}{m-j}}{\binom{F}{n}}
\end{equation}
Here $F$ is the maximum possible number of edges in the graph, that is 
$F = (n^2-n)/2$, while $M$ is the maximum number of edges within a 
cluster $A$ of size $k$, that is $M = (k^2-k)/2$. $p$ is the actual 
number of links found in the community. The authors have not proposed a 
similar formula for weighted graphs.

An alternative method of testing graph clustering algorithms is to 
design a series of graphs with known clustering and see how well the 
algorithm in question predicts these graphs. The most famous of such 
graphs was introduced by \cite{girvan2002}. They consisted of 128 
vertices with four clusters of even size. The average degree of each 
vertix was 16. A parameter $k_{out}$ would designate how many edges a 
node would have leading to other clusters. With $k_{out} < 8$ a good 
clustering algorithm should be expected to find the original clustering.  
However these graphs are a very limited subset of the real graphs that 
clustering algorithms are supposed to work with. In particular it's 
small, the communities are of an even size and the nodes have very 
similar degrees\footnote{As pointed out by \cite{lancichinetti2008}}. 
These shortcomings lead to misleading benchmark results for algorithms 
that are not suited to work on for example graphs with clusters of very 
different sizes.

To improve on these shortcomings, another group of random graphs was 
introduced in 2009 by \cite{lancichinetti2008}. With the implicit 
assumption that degree and community size in real graphs are guided by 
power laws, they device a set of graphs usually denominated as 
\emph{LFR} graphs where each node is given a degree taken from a power 
law distribution with exponent $\gamma$ such that the average degree is 
$k$.  Each node keeps a fraction $1-\mu$ of its edges within its own 
cluster and the remaining fraction $\mu$ with nodes outside its cluster.  
The sizes of clusters are also taken from a power law distribution. The 
random graphs are then constructed by assigning the nodes randomly to 
clusters as long as each node has less inter-cluster links than the size 
of the cluster. The graph is then adjusted to make sure the amount of 
inter-cluster versus intra-cluster links is given according to $\mu$.

The \emph{LFR} graphs are like their predecessors the \emph{GN} graphs a 
limited benchmark in the sense that they will reward algorithms that are 
particularly good at clustering graphs that correspond to the particular 
structure determined by the power distribution of of edges and cluster 
sizes.

These graphs has later been used by \cite{lancichinetti2009} in a 
benchmark of 12 clustering algorithms, many of which are mentioned in 
the following section of this paper. The \emph{LFR} are also used by 
\cite{aldecoa2010} to show that \emph{Surprise} is better correlated 
with the true clustering than \emph{Modularity} in a series of tests 
with various algorithms.

\subsection{Girvan Newman clustering}
If we look at the problem of clustering a graph as a problem of removing 
edges until we arrive at a set amount of communities, we need a good 
measure for what edges should be removed. The Girvan-Newman algorithm 
(in short \emph{GN}) proposes that we remove edges based on a measure of 
``betweenness'', that is how much a particular edge $e_{ij}$ is situated 
between clusters. The betweenness of $e_{ij}$ is calculated by counting 
how many shortest paths between pairs of vertices that run through 
$e_{ij}$. In case there is ($k > 1$) shortest paths, each path 
traversing $e_{ij}$ contribues $\frac{1}{k}$ to the count.  If we define 
$\sigma_{st}$ as the number of shortest paths between vertic $s$ and 
$t$, and $\sigma_{st}(v)$ as the number of shortest paths between $s$ 
and $t$ traversing $e_{ij}$ we can write betweenness as
\begin{equation}
	C_B(v)= \sum_{s \neq v \neq t \in V} \
	\frac{\sigma_{st}(v)}{\sigma_{st}}
\end{equation}
Using this measure we can iteratively construct a dendrogram by 
calculating all the shortest paths of the graph, then finding the edge 
with the highest betweenness and finally remove this edge and repeat. A 
common way to find the best amount of groupings is to stop removing 
edges when the highest value measure of \emph{Community} is obtained.
The complexity of this algorithm is $O(n^3)$ in the general case, but it
can be calculated slightly faster when the graph is sparse. The 
algorithm does not perform particularly well on random graphs, but it is 
very simple to implement and requires no additional parameters.


% https://en.wikipedia.org/wiki/Girvan%E2%80%93Newman_algorithm
% girvan2002

\subsection{Spectral clustering}
From spectral graph theory we get the spectral clustering algorithm 
which in reality is an ensemble of different algorithms all solving the 
problem of clustering a similarity graph in similar ways. The general 
idea is to use the eigenspectrum of the graph Laplacian to derive a set 
of $k$ othonormal vectors that are trivial to cluster using a k-means 
algorithm to arrive at $k$ clusters\cite{von2007}. The parameter $k$ 
needs to be decided in advance. The differences across spectral 
clustering algorithms usually lie in how the graph Laplacian is defined.  
One approach is to define the Laplacian $L_G$ for a graph $G$ as $L_G = 
D_G - A_G$. However it has been shown that using the normalized 
Laplacian yields better results.  One way to define the normalized 
Laplacian as used in \cite{ng2002} is $L_G = I - D_{G}^{-1/2} A_G 
D_{G}^{-1/2}$.

Spectral clustering suffers from the fact that we need to decide the 
amount of clusters to output before running the algorithm. This can be 
avoided by calculating clusterings for $2$ to $k$ groupings and then 
find the number that optimizes a given measure of quality, for example 
the measure of community. The complexity of calculating the eigenvectors 
is $O(n^3)$ in the worst case. This can be sped up using the Lanczos 
method\footnote{See for example \cite{golub1996}}, but the worst case 
complexity is still the same. As for the quality of a solution derived 
with spectral clustering a lot of research has been made on the subject, 
but there has not been any definite results. In \cite{lancichinetti2009} 
a spectral method was compared to several other algorithms using a 
randomly generated graph and not found particularly well performing.  A 
big advantage of spectral clustering is that it is simple to implement.
% ng2002 the algorithm I use
% von2007 tutorial

\subsection{An approach from Information Theory}
Instead of looking at a graph theoretic measures of clusterness, an 
alternative idea proposed by \cite{rosvall2008} is to turn to 
Information Theory. The approach they propose is to look at clustering 
as a problem of how best to encode the the path of a random walk. A 
random walk in a weighted graph is a traversal from node to node where a 
step from a vertix $v_i$ to another vertix $v_j$ connected to $v_i$ by 
the edge $e_{ij}$ happens with a probability $p_{ij}$. We can calculate 
$p_{ij}$ as $\frac{w_{ij}}{d_i}$. This means that edges with a high 
weight are traveled often and nodes with a high degree are visited 
often. If we choose to characterize the random walk as a walk that most 
of the time remains within clusters and occasionally goes between 
clusters we can encode this walk by attributing each cluster a word, and 
then for each cluster reuse the same codewords. This way we can derive 
the average description length of a single step as follows:
\begin{equation}
	L(\textbf{M}) = q H(\mathcal{Q}) + \sum_{i=1}^{m} p^i \
	H(\mathcal{P}^i)
\end{equation}
In this equation $H(\mathcal{Q})$ is the entropy of the cluster names 
and $H(\mathcal{P}^i)$ is the entropy of the within-cluster movements 
including an extra word to designate that we are leaving the cluster.  
$q$ is the probability that we switch between clusters at any given 
step, while $p$ is the fraction of within-cluster movements that occur 
in cluster $i$ plus the probability of exiting cluster $i$. Optimizing 
this measure leads us to an optimal clustering. In practice this is done 
by exploring the space of possible clusters using a breadth first search 
and refining these results using simulated annealing.

This approach has several positive properties. It doesn't need 
parameters and is relatively fast with a complexity of $O(m)$ and in 
\cite{lancichinetti2009} it is rated as one of the best performing 
algorithms. However the implementation is relatively involved using 
several different concepts stitched together. This algorithm is labeled 
\emph{Infomod} when referenced later.
% rosvall2008

\subsection{The Louvain Method}
The \emph{Louvain} Method was introduced in 2008 by \cite{blondel2008} 
and made it possible to feasibly cluster more than 100 million nodes as 
compared to the few million nodes which had formerly been possible with 
the fastest methods. The method is based on the maximizing the 
\emph{Modularity} of a graph iteratively. This is done efficiently 
because the gain in modularity from moving a vertex $v_i$ to a community 
$C$ can be calculated easily:
\begin{multline}
	\Delta Q = \left[ \frac{\sum_{in} + 2 k_{i,in}}{2m} - \
	\left(\frac{\sum_{tot} + k_i}{2m} \right)^2 \right] \\
	- \left[\frac{\sum_{in}}{2m} - \left(\frac{\sum_{tot}}{2m} \
	\right)^2 - \left( \frac{k_i}{2m} \right)^2 \right]
\end{multline}
Here as in the definition of \emph{Modularity}, $m=\frac{1}{2} \sum_{ij} 
A_{ij}$, $k_i = \sum_j A_{ij}$ and $k_{i,in}$ is the sum of weights of 
the edges from $v_i$ to nodes in $C$.  $\sum_{in}$ is the sum of the 
weights of the edges of $C$ while $\sum_{tot}$ is the sum of all edges 
with at least one vertex in $C$.  The algorithm works by assigning each 
node to it's proper cluster and then for each iteration reassign nodes 
to neighboring communities if it increases the community measure.  These 
iterations are carried out until there are no reassignments in which 
case an initial clustering has been reached.  For graphs consisting of a 
large number of nodes, the algorithm can now be repeated, treating each 
of the resulting clusters as a node and repeating the same procedure.

In terms of speed and simplicity the \emph{Louvain} method is one of the 
best methods available today. It has a computational complexity of 
$O(m)$ and is very simple to implement. What's more, there is no need to 
specify how many clusters we should receive and the individual cluster 
sizes have shown to be sensible in real world trials.  As for the 
quality of the solution it compares favorably to other clustering 
methods in the comparison on \emph{LFR} graphs made by 
\cite{lancichinetti2009}.
% blondel2008


% \subsection{Expectation minimization clustering}
% leicht2008
% not suitable since it works in directed graphs
\subsection{Potts Model Approach}
Inspired by the \emph{Louvain} Method an algorithm based on Potts model 
was introduced in 2009 by \cite{ronhovde2009}. Instead of working with 
the community measure it is based on Potts Model Hamiltonian which is 
defined as follows:
\begin{equation}
	\mathcal{H}(\{c\}) = \frac{1}{2} \sum_{i \neq j} (A_{ij} - \
	\gamma \mathcal{1}_{\{A_{ij} = 0\}} \delta(c_i, c_j)
\end{equation}
$\delta(c_i, c_j)$ is the Kroneker Delta and is 1 only if the community 
assignment of $v_i$, $c_i$ is equal to the community assignment of 
$v_j$, $c_j$. $\gamma$ is a parameter specified before the algorithm is 
run. As with the \emph{Louvain} method each node is assigned to its 
proper cluster and then iteratively reassigned to neighboring clusters 
based on the above measure. Different from the \emph{Louvain} method, 
this assignment is done several times and the best cluster is picked 
out.  Given to partitions $A$ and $B$ this is done with measures on the 
entropy ($H(A)$ and $H(B)$), the mutual information ($I(A,B)$), the 
normalized mutual information ($I_N(A,B)$) and the variation of 
information ($V(A,B)$).  For the graph they are defined as follows:
\begin{eqnarray}
	H(A) & = & - \sum_{k=1}^{q_A} \frac{n_k}{n} log \frac{n_k}{N} \\
	I(A,B) & = & \sum_{i=1}^{q_A} \sum_{j=1}^{q_B} \frac{n_{ij}}{N} \
	log \frac{n_{ij}N}{v_i v_j} \\
	I_N(A,B) & = & \frac{2I(A,B)}{H(A) + H(B)} \\
	V(A,B) & = & H(A) + H(B) - 2I(A,B)
\end{eqnarray}
Here $q_A$ is the amount of clusters in $A$ where as $n$ is the total 
amount of vertices and $n_k$ the total amount of vertices in cluster $k$ 
of partition $A$. $n_{ij}$ is the amount of nodes shared by cluster $i$ 
of partition $A$ and cluster $j$ of partition $B$. For each run, the 
gamma parameter is adjusted and another clustering is calculated. Then 
the best clustering is picked by looking at the regions of the graph 
that are strongly correlated measured by high $I_N$ and low $V$.
The algorithm performs well according to \cite{lancichinetti2009} and is 
very fast with a complexity of $O(m^\beta log n)$, $\beta \approx 1.3$.  
It is however complex to implement and requires that several model 
parameters are supplied. In the following this algorithm is referred to 
as \emph{Potts}.

\subsection{An overview}

\begin{table}
	\small
\begin{tabular}{l*{4}{c}}
	Author(s) & Year & Label & Order \\
	\hline
	\noalign{\smallskip} 
	Girvan and Newman & 2002 & \emph{GN} & $O(n^3)$ \\
	Ng \& al. & 2002 & \emph{Spectral} & $O(n^3)$ \\
	Rosvall and Bergstrom & 2008 & \emph{Infomap} & $O(m)$ \\
	Blondel \& al. & 2008 & \emph{Louvain} & $O(m)$ \\
	Ronhovede and Nussinov & 2009 & \emph{Potts} & $O(m^\beta log n)$ \\
\end{tabular}
\caption{Complexity Overview}
\end{table}

As seen in the complexity overview, the differences between the fastest 
and slowest algorithms are vast. With more than a few thousand vertices, 
calculating the spectral clustering or using the Girvan-Newman method 
will be unfeasible on most hardware. The quality of each algorithm as 
measured by benchmarking them on the \emph{LFR} graphs show that the 
three best algorithms, \emph{Louvain}, \emph{Potts} and \emph{Infomap} 
perform well on this type of random graphs, with \emph{Infomap} 
performing better on larger graphs\footnote{A much more detailed 
analysis of these results can be found in \cite{lancichinetti2009}}. 
