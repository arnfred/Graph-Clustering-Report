\section{Related work}

Over the last 40 years a large variety of algorithms have been invented 
to cluster graphs. Most of these methods were based on different 
measures of optimal clustering all of which were usually np-hard to 
derive optimal solutions for. The field has seen a boost in recent years 
due to the increased demands for handling large data that are usually 
neatly abstracted as graphs\cite[p. 2]{fortunato2010}. Examples include 
social networks and the world wide web.

In this paper I don't intend to do a thorough review of the field, 
something which many much better abled people have already done before 
me\footnote{Examples include \cite{newman2004}, \cite{shcaeffer2007} and 
\cite{fortunato2010}}. Instead I have tried to pick out a few distinct 
algorithms that are good enough that each could have been a candidate 
for a clustering algorithm used for grouping the scientific articles.


% von2007 : schaeffer
% fortunato2010 : community detection in graphs
% lancichinetti2008 : comparison
% newman2005 : short and old overview

\subsection{Introduction of related work} % delete this header later

\subsection{How to measure clustering}

\subsection{Minimum-cut clustering}
% newman2005 and probably von2007

\subsection{Girvan–Newman clustering}
% https://en.wikipedia.org/wiki/Girvan%E2%80%93Newman_algorithm
% girvan2002

\subsection{Spectral clustering}
% ng2002 the algorithm I use
% von2007 tutorial

\subsection{Random Walk clustering
% rosvall2008

\subsection{Community maximization clustering}
The ``Louvain Method'' as it is popularly called was introduced in 2008 
by \cite{blondel2008} and made it possible to feasibly cluster more than 
100 million nodes as compared to the few million nodes which had 
formerly been possible with the fastest methods. The method is based on 
the measure of modularity in a graph which measures the density of links 
inside communities compared to the links between communities as a scalar 
between $1$ and $-1$\cite{girvan2002}. For a weighted graph it is 
defined as follows
\begin{equation}
	Q = \frac{1}{2m} \sum_{i,j} \leftbracket A_{ij} - \frac{k_i k_j}{2m} 
	\rightbracket \delta(c_i, c_j)
\end{equation}
where $m=\frac{1}{2} \sum_{ij} A_{ij}$, $k_i = \sum_j A_{ij}$ and 
$A_{ij}$ is the weight between the edge connecting $i$ and $j$. The 
algorithm works by assigning each node to it's proper cluster and then 
for each iteration reassign nodes to neighboring communities if it 
increases the community measure. These iterations are carried out until 
there are no reassignments in which case an initial clustering has been 
reached. For graphs consisting of a large number of nodes, the algorithm 
can now be repeated, treating each of the resulting clusters as a node 
and repeating the same procedure.

In terms of speed and simplicity the Louvain method is one of the best 
methods available today. It has a computational complexity of TODO [add 
complexity] and is very simple to implement. What's more, there is no 
need to specify how many clusters we should receive and the individual 
cluster sizes have shown to be sensible in real world trials. As for the 
quality of the solution it compares favorably to other clustering 
methods in a comparison in \cite{lancichinetti2008} only beaten by WHAT 
AND WHAT?.
% blondel2008

\subsection{Expectation minimization clustering}
% leicht2008
