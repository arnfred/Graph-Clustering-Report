\section{Related work}

Over the last 40 years a large variety of algorithms have been invented 
to cluster graphs. Most of these methods were based on different 
measures of optimal clustering all of which were usually np-hard to 
derive optimal solutions for. The field has seen a boost in recent years 
due to the increased demands for handling large data that are usually 
neatly abstracted as graphs\cite[p. 2]{fortunato2010}. Examples include 
social networks and the world wide web.

In this paper I don't intend to do a thorough review of the field, 
something which many much better abled people have already done before 
me\footnote{Examples include \cite{newman2004}, \cite{schaeffer2007} and 
\cite{fortunato2010}}. Instead I have tried to pick out a few distinct 
algorithms that are good enough that each could have been a candidate 
for a clustering algorithm used for grouping the scientific articles.


% von2007 : schaeffer
% fortunato2010 : community detection in graphs
% lancichinetti2008 : comparison
% newman2005 : short and old overview

\subsection{Introduction of related work} % delete this header later

\subsection{How to measure clustering}

\subsection{Minimum-cut clustering}
% newman2005 and probably von2007

\subsection{Girvan–Newman clustering}
If we look at the problem of clustering a graph as a problem of removing 
edges until we arrive at a set amount of communities, we need a good 
measure for what edges should be removed. The Girvan-Newman algorithm 
proposes that we remove edges based on a measure of ``betweenness'', 
that is how much a particular edge $e_{ij}$ is situated between 
clusters. The betweenness of $e_{ij}$ is calculated by counting how many 
shortest paths between pairs of vertices that run through $e_{ij}$. In 
case there is ($k > 1$) shortest paths, each path traversing $e_{ij}$ 
contribues $\frac{1}{k}$ to the count.  If we define $\sigma_{st}$ as 
the number of shortest paths between vertic $s$ and $t$, and 
$\sigma_{st}(v)$ as the number of shortest paths between $s$ and $t$ 
traversing $e_{ij}$ we can write betweenness as
\begin{equation}
	C_B(v)= \sum_{s \neq v \neq t \in V} \
	\frac{\sigma_{st}(v)}{\sigma_{st}}
\end{equation}
Using this measure we can iteratively construct a dendrogram by 
calculating all the shortest paths of the graph, then finding the edge 
with the highest betweenness and finally remove this edge and repeat. A 
common way to find the best amount of groupings is to stop removing 
edges when the highest value measure of community\footnote{This measure 
	is further explained when reviewing the Louvain method as well as in 
the chapter on measures of graph quality} is obtained.
The complexity of this algorthim is $O(n^3)$ in the general case, but it
can be calculated slightly faster when the graph is sparse. The 
algorithm does perform particularly well on random graphs, but it is 
very simple to implement and requires no additional parameters.


% https://en.wikipedia.org/wiki/Girvan%E2%80%93Newman_algorithm
% girvan2002

\subsection{Spectral clustering}
From spectral graph theory we get the spectral clustering algorithm 
which in reality is an ensemble of different algorithms all solving the 
problem of clustering a similarity graph in similar ways. The general 
idea is to use the eigenspectrum of the graph Laplacian to derive a set 
of $k$ othonormal vectors that are trivial to cluster using a k-means 
algorithm to arrive at $k$ clusters\cite{von2007}. The parameter $k$ 
needs to be decided in advance. The differences across spectral 
clustering algorithms usually lie in how the graph Laplacian is defined.  
One approach is to define the Laplacian $L_G$ for a graph $G$ as $L_G = 
D_G - A_G$. However it has been shown that using the normalized 
Laplacian yields better results.  One way to define the normalized 
Laplacian as used in \cite{ng2002} is $L_G = I - D_{G}^{-1/2} A_G 
D_{G}^{-1/2}$.

Spectral clustering suffers from the fact that we need to decide the 
amount of clusters to output before running the algorithm. This can be 
avoided by calculating clusterings for $2$ to $k$ groupings and then 
find the number that optimizes a given measure of quality, for example 
the measure of community. The complexity of calculating the eigenvectors 
is $O(n^3)$ in the worst case. This can be sped up using the Lanczos 
method\footnote{See for example \cite{golub1996}}, but the worst case 
complexity is still the same. As for the quality of a solution derived 
with spectral clustering a lot of research has been made on the subject, 
but there has not been any definite results. In \cite{lancichinetti2009} 
a spectral method was compared to several other algorithms using a 
randomly generated graph and not found particularly well performing.  A 
big advantage of spectral clustering is that it is simple to implement.
% ng2002 the algorithm I use
% von2007 tutorial

\subsection{An approach from Information Theory}
Instead of looking at a graph theoretic measures of clusterness, an 
alternative idea proposed by \cite{rosvall2008} is to turn to 
Information Theory. The approach they propose is to look at clustering 
as a problem of how best to encode the the path of a random walk. A 
random walk in a weighted graph is a traversal from node to node where a 
step from a vertix $v_i$ to another vertix $v_j$ connected to $v_i$ by 
the edge $e_{ij}$ happens with a probability $p_{ij}$. We can calculate 
$p_{ij}$ as $\frac{w_{ij}}{d_i}$. This means that edges with a high 
weight are traveled often and nodes with a high degree are visited 
often. If we choose to characterize the random walk as a walk that most 
of the time remains within clusters and occasionally goes between 
clusters we can encode this walk by attributing each cluster a word, and 
then for each cluster reuse the same codewords. This way we can derive 
the average description length of a single step as follows:
\begin{equation}
	L(\textbf{M}) = q H(\mathcal{Q}) + \sum_{i=1}^{m} p^i \
	H(\mathcal{P}^i)
\end{equation}
In this equation $H(\mathcal{Q})$ is the entropy of the cluster names 
and $H(\mathcal{P}^i)$ is the entropy of the within-cluster movements 
including an extra word to designate that we are leaving the cluster.  
$q$ is the probability that we switch between clusters at any given 
step, while $p$ is the fraction of within-cluster movements that occur 
in cluster $i$ plus the probability of exiting cluster i. Optimizing 
this measure leads us to an optimal clustering. In practice this is done 
by exploring the space of possible clusters using a breadth first search 
and refining these results using simulated annealing.

This approach has several positive properties. It doesn't need 
parameters and is relatively fast with a complexity of $O(m)$ and in 
\cite{lancichinetti2009} it is rated as one of the best performing 
algorithms. However the implementation is relatively involved using 
several different concepts stitched together.
% rosvall2008

\subsection{Community maximization clustering}
The ``Louvain Method'' as it is popularly called was introduced in 2008 
by \cite{blondel2008} and made it possible to feasibly cluster more than 
100 million nodes as compared to the few million nodes which had 
formerly been possible with the fastest methods. The method is based on 
the measure of modularity in a graph which measures the density of links 
inside communities compared to the links between communities as a scalar 
between $1$ and $-1$\cite{girvan2002}. For a weighted graph it is 
defined as follows
\begin{equation}
	Q = \frac{1}{2m} \sum_{i,j} \lbrack A_{ij} - \frac{k_i k_j}{2m} 
	\rbrack \delta(c_i, c_j)
\end{equation}
where $m=\frac{1}{2} \sum_{ij} A_{ij}$, $k_i = \sum_j A_{ij}$ and 
$A_{ij}$ is the weight between the edge connecting $i$ and $j$. The 
algorithm works by assigning each node to it's proper cluster and then 
for each iteration reassign nodes to neighboring communities if it 
increases the community measure. These iterations are carried out until 
there are no reassignments in which case an initial clustering has been 
reached. For graphs consisting of a large number of nodes, the algorithm 
can now be repeated, treating each of the resulting clusters as a node 
and repeating the same procedure.

In terms of speed and simplicity the Louvain method is one of the best 
methods available today. It has a computational complexity of TODO [add 
complexity] and is very simple to implement. What's more, there is no 
need to specify how many clusters we should receive and the individual 
cluster sizes have shown to be sensible in real world trials. As for the 
quality of the solution it compares favorably to other clustering 
methods in a comparison in \cite{lancichinetti2009} only matched by the 
so-called InfoMap algorithm proposed in \cite{rosvall2008} (described 
above) and the algorithm proposed in \cite{ronhovde2009}.
% blondel2008

\subsection{Expectation minimization clustering}
% leicht2008
